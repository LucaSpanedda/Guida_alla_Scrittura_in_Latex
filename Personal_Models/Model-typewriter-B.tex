% LATEX TYPEWRITER DOCUMENT MODEL B --------------------------- 80 CHARACTERS :

% ---------------------------------------------------------------- PARAMETERS :
\documentclass{article}
\usepackage{manuscript,lipsum}

\begin{document}

% TITLE OF THE TEXT
\title{Paper di prova MACCHINA DA SCRIVERE B}
\author{Luca Spanedda}
% print date of today
\date{\today}
% make the title here
\maketitle

% ABSTRACT
\begin{abstract}
I segnali caotici nella sintesi digitale possono essere utili per implementare  
oscillatori e/o segnali di controllo tempo-varianti.
Un problema dei sintetizzatori digitali e degli effetti audio,
è che sono spesso caratterizzati da suoni molto lontani da quelli prodotti nel mondo fisico,
a causa della natura precisa e tempo-invariante della loro generazione del segnale nel mondo digitale.
\newline
Nel computer, i dettagli che in natura nel suono si verificano normalmente in modo imprevedibile, 
devono essere accuratamente sequenziati fino al punto di esaurimento 
delle risorse del computer e del programmatore.
\newline
Adottare segnali caotici nella sintesi e nel controllo, può essere dunque un modo
per produrre suoni più naturali di quelli generati attraverso le tecniche più standard di sintesi digitale,
con metodi computazionali più economici.
\newline
In questo studio andremo ad implementare dei circuiti nel linguaggio di programmazione Faust(GRAME)
per rappresentare discretamente alcuni modelli caotici, ed alcuni stocastici,
che possano essere utili nella generazione di texture sonore di sintesi.
Estrapoleremo da questi codici alcune topologie per rappresentare i circuiti corrispondenti
alle equazioni differenziali implementate, ed andremo ad utilizzare il linguaggio di programmazione 
Python per ottenere alcuni plot che rappresentino il comportamento di questi modelli.
\end{abstract}

\pagebreak
\lipsum[1-3]

\end{document}