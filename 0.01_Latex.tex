% per commentare in Latex:
% usare %, che prende in considerazione la riga su cui è scritto il testo
% backslash \ o barra rovescia, introduce i comandi.
% Non è necessario il ; alla fine delle istruzioni.

% le prime istruzioni da scrivere in Latex sono:
% Il preambolo, è la parte iniziale del documento,
% che comprende la classe del documento con le sue specifiche
% dimensione del carattere, scelto qui di 12pt di dimensione,
% e tra le parentesi {} il tipo di documento che si sta scrivendo
% può essere:

% article
% è una classe progettata per scrivere articoli – non articoli di giornale,
% ma su riviste specializzate! – e quindi non prevede l’uso del comando
%\ chapter, in quanto la suddivisione è solo in paragrafi e sottoparagrafi.

% book 
% serve a scrivere documenti più lunghi e complessi, che contengano una
% suddivisione in sezioni più articolata e comprensiva di parti e capitoli.

% letter 
% è una classe pensata appositamente per la composizione di lettere.

% beamer 
% è la classe per le presentazioni – slides – di cui però parleremo più
% diffusamente in seguito, perché ha delle caratteristiche molto particolari.

% Ce ne sono molte altre, come thesis per le tesi di laurea 
% e report, simile a book, la cui trattazione però va oltre gli scopi di questa guida, 
% che vuole essere introduttiva.

\documentclass[12pt,a4paper]{book}


% Subito dopo la dichiarazione della classe del documento, bisogna indicare i
% pacchetti che si vogliono usare nel progetto. Il comando da dare in questo caso 
% è \usepackage:

\usepackage[italian]{babel}



% A questo punto, diamo il comando che
% segnala la fine del preambolo e l’inizio del documento vero e proprio:
\begin{document}

% Con \title dite a LATEX qual è il titolo del vostro libro, e questo titolo verrà
% scritto nel punto in cui avete inserito il comando \maketitle.
% titolo del testo, che viene visualizzato

\title{Paper di prova}
% Autore del testo, che viene visualizzato
\author{Luca Spanedda}
% Data
\date{}
% ...
\maketitle


\chapter{Titolo del capitolo}
% Titolo del capitolo
\section{Titolo del paragrafo}
% Titolo del paragrafo
\subsection{Titolo del sottoparagrafo}
% Titolo del sottoparagrafo


%se volete iniziare un nuovo capoverso,
%dovete lasciare una riga vuota. Se vi limitate semplicemente ad andare a capo,
%LATEX interpreterà la cosa come un semplice spazio e quindi non inizierà un
%nuovo capoverso!

% la fine del documento vero e proprio:
\end{document}