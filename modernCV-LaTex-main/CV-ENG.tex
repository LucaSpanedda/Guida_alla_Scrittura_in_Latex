\documentclass[10pt,a4paper,sans]{moderncv}
\moderncvstyle{classic}
\moderncvcolor{black}

\usepackage[italian]{babel}
\usepackage[utf8]{inputenx}
\usepackage[left=1cm,right=1cm,top=1.2cm,bottom=1.2cm]{geometry}

% questa riga allarga la colonna di sinistra
\setlength{\hintscolumnwidth}{3.7cm}

% dati personali
\firstname{Luca}
\familyname{Spanedda}
\title{Curriculum Vitae}
\address{Via Parigi 58}{00055, Ladispoli(RM), Italy}
\mobile{+39\,3663759122}
\email{lucaspanedda1995@gmail.com}
\homepage{lucaspanedda.github.io/} 

\begin{document}

\maketitle

\section{Personal information}
\cvline{first name}{Luca}
\cvline{last name}{Spanedda}
\cvline{place and date of birth}{Rome (Italy), 15-02-1995}
\cvline{nationality}{Italian}

\section{Education}
\cventry{2015}{Second grade secondary school Diploma (Ladispoli(RM), Italy)}{}{}{}{
Second grade secondary school Diploma, ITC Technical and commercial institute
achieved at ISIS Giuseppe Di Vittorio, Ladispoli 00055 (RM), Italia.\medskip
}

\cventry{2020}{Three-year degree in New technologies and musical languages - Electronic Music (Rome, Italy)}{}{}{}{Three-year degree achieved at Santa Cecilia Conservatory (RM), Italia.\medskip
}

\cventry{2020}{Master's degree student in New technologies and musical languages - Electronic Music (Rome, Italy)}{}{}{}{at Santa Cecilia Conservatory (RM), Italia.\medskip
}

\section{Experiences and Skills}
\cvline{-}{Theory, rhythm and musical perception (solfeggio, ear training)}
\cvline{-}{Compositional techniques: counterpoint, choral, contemporary}
\cvline{-}{Musical Instrumental and Electroacoustic Composition}
\cvline{-}{Basic experience with piano and keyboards; advanced with synthesizers}
\cvline{-}{Computer Music and Computer Science (DSP) with: Python, C, Arduino, LaTex, Lilypond, Pure Data, Max Msp, Faust, CSound, ecc.}
\cvline{-}{Computer Science and skills with: C, LaTex, Lilypond. Shell: Bash, CMD, ecc.}
\cvline{-}{Musical Programming Microprocessors/Microcontrollers: Teensy, Arduino, Raspberry, ecc.}
\cvline{-}{Mathematics of sounds, Acoustics and Psychoacoustics; analysis of environments and musical instruments (Music information retrieval - MIR)}
\cvline{-}{Basic experience in electrical engineering: Audio circuits, Analog synthesizers, Audio processing}
\cvline{-}{Sound engineer, Mixing, Mastering. DAWs: Reaper, Ableton Live}
\cvline{-}{Multimedia composition: Jitter(Max) and Processing.}
\cvline{-}{History of Music and History of Electroacoustic Music; Musical analysis}
\cvline{2012 - current period}{Music production: various LPs for independent labels; Live performances.}
\cvline{2013}{Realization of a piece used for the soundtrack of the film “Aquadro” by Stefano Lodovichi produced by Rai cinema. Project for the composition and development of a piece, at the Art Music Recording Studios in Gallarate (VA, Italy)}
\cvline{2017}{Presentation and performance of Audio-Visual composition at: Master in Sonic Arts University of Rome "Tor Vergata", and Santa Cecilia Conservatory}
\cvline{2021}{Performance of the Live Electronics part of Post-prae-ludium n. 1 for Donau - Luigi Nono; at: Art and Science Festival at the Goethe-Institut Italien in Rome}
\cvline{2021}{Presentation and Live Electronics performance of the LP "Particles" at: Klang Roma (Pigneto, Rome)}
\cvline{2021}{Realization of a piece used for the soundtrack of the film “From my house in da house” by Giovanni La Gorga.}

\cleardoublepage 

\section{Seminars and training meetings}
\cvline{2020}{Participation in Online Class "Build Soundart Devices: Powclass Nº6" with Moon Armada held by: Powland}
\cvline{2020}{Participation in Webinar "Faust 101 for the confined" held by: Grame}
\cvline{2021}{Participation in Workshop "Faust Physical Modeling Workshop" held by: Grame}
\cvline{2021}{Participation in Webinar "Alla scoperta di Arduino" held by: WiFi Informatica}
\cvline{2021}{Seminar participation at Auditorium Parco della Musica (RM) held by: 
Fondazione Musica per Roma “Dal segno al suono: il lavoro del musicista”. On the occasion of the PMCE concert – Parco della Musica Contemporanea Ensemble: "VARIAZIONI DI LUCE" - Grisey, Lupone}
\cvline{2021}{Seminars participation at Conservatorio Santa Cecilia (RM) held by: 
Giorgio Netti, Giorgio Nottoli, Simone Pappalardo. }

\section{Languages}
\cvline{Italian}{native language}
\cvline{English}{fluent}

\section{Links}
\cvline{1}{Publications on Academia: https://conservatoriosantacecilia.academia.edu/LucaSpanedda}
\cvline{2}{Codes/Research: https://github.com/LucaSpanedda/}

\begin{scriptsize}
\vspace{\fill} 
Rome, \today
\end{scriptsize}

\begin{footnotesize}
\begin{Verbatim}
I authorize the processing of my personal data present in the curriculum vitae pursuant to Legislative Decree 30 June 2003, n. 196 and of the GDPR (EU Regulation 2016/679).
\end{Verbatim}
\end{footnotesize}

\end{document}