\documentclass[10pt,a4paper,sans]{moderncv}
\moderncvstyle{classic}
\moderncvcolor{black}

\usepackage[italian]{babel}
\usepackage[utf8]{inputenx}
\usepackage[left=1cm,right=1cm,top=1.2cm,bottom=1.2cm]{geometry}

% questa riga allarga la colonna di sinistra
\setlength{\hintscolumnwidth}{3.7cm}

% dati personali
\firstname{Luca}
\familyname{Spanedda}
\title{Curriculum Vitae}
\address{Via Parigi 58}{00055, Ladispoli(RM), Italy}
\mobile{+39\,3663759122}
\email{lucaspanedda1995@gmail.com}
\homepage{lucaspanedda.github.io/} 

\begin{document}

\maketitle

\section{Personal information}
\cvline{first name}{Luca}
\cvline{last name}{Spanedda}
\cvline{place and date of birth}{Rome (Italy), 15-02-1995}
\cvline{nationality}{Italian}

\section{Education}
\cventry{2015}{Diploma di Scuola Secondaria di II grado (Ladispoli(RM), Italy)}{}{}{}{Diploma di Scuola Secondaria di II grado, Tecnico Commerciale
conseguito presso ISIS Giuseppe Di Vittorio, Ladispoli 00055 (RM), Italia.\medskip
}

\cventry{2020}{Laurea Triennale in Nuove tecnologie e linguaggi musicali - Musica Elettronica (Roma, Italy)}{}{}{}{Laurea conseguita presso Conservatorio Santa Cecilia (RM), Italia.\medskip
}

\cventry{2020}{Studente Magistrale in Nuove tecnologie e linguaggi musicali - Musica Elettronica (Roma, Italy)}{}{}{}{presso Conservatorio Santa Cecilia (RM), Italia.\medskip
}

\section{Experiences and Skills}
\cvline{-}{Teoria, ritmica e percezione musicale (solfeggio, ear training)}
\cvline{-}{Tecniche compositive: contrappunto, corale, contemporanea}
\cvline{-}{Composizione musicale Strumentale ed Elettroacustica}
\cvline{-}{Esperienze con pianoforte, tastiere, sintetizzatori}
\cvline{-}{Programmazione e Informatica Musicale (DSP): Python, C, Arduino, LaTex, Lilypond, Pure Data, Max Msp, Faust, CSound, ecc.}
\cvline{-}{Programmazione e competenze informatiche: C, LaTex, Lilypond. Shell: Bash, CMD, ecc.}
\cvline{-}{Programmazione Musicale Microprocessori/controllori: Teensy, Arduino, Raspberry, ecc.}
\cvline{-}{Matematica, Acustica e Psicoacustica; analisi degli Ambienti e degli Strumenti Musicali}
\cvline{-}{Esperienze in elettrotecnica: Circuiti audio, Sintetizzatori analogici, Audio processing}
\cvline{-}{Fonia, Mixing, Mastering. DAWs: Reaper, Ableton Live}
\cvline{-}{Composizione multimediale: Jitter(Max) e Processing.}
\cvline{-}{Storia della musica e Storia della musica Elettroacustica; Analisi musicale}
\cvline{2012 - periodo corrente}{Produzione musicale: vari LP per etichette indipendenti; Live performances.}
\cvline{2013}{Realizzazione di un brano utilizzato per la colonna sonora del film “Aquadro” di Stefano Lodovichi prodotto da Rai cinema. Progetto di composizione e sviluppo di un brano, presso l’Art Music Recording Studios di Gallarate (VA, Italy)}
\cvline{2017}{Presentazione ed esecuzione di composizione Audio-Visuale presso: Master inSonic Arts Università degli studi di Roma “Tor Vergata”, e Conservatorio Santa Cecilia}
\cvline{2021}{Esecuzione della parte Live Electronics di Post-prae-ludium n. 1 per Donau - Luigi Nono; presso: Festival Arte e Scienza al Goethe-Institut Italien di Roma}
\cvline{2021}{Presentazione ed esecuzione Live Electronics di "Particles" LP presso: Klang Roma (Pigneto, Roma)}
\cvline{2021}{Realizzazione di un brano utilizzato per la colonna sonora del film “From my house in da house” di Giovanni La Gorga.}

\cleardoublepage 

\section{Seminars and training meetings}
\cvline{2020}{Partecipazione a Classe Online "Build Soundart Devices: Powclass Nº6" con Moon Armada - Powland}
\cvline{2020}{Partecipazione Webinar "Faust 101 for the confined" - Grame}
\cvline{2021}{Partecipazione Workshop "Faust Physical Modeling Workshop" - Grame}
\cvline{2021}{Partecipazione Webinar "Alla scoperta di Arduino" - WiFi Informatica}
\cvline{2021}{Partecipazione a Seminario presso Auditorium Parco della Musica (RM) tenuto da: 
Fondazione Musica per Roma “Dal segno al suono: il lavoro del musicista”. In occasione del concerto del PMCE – Parco della Musica Contemporanea Ensemble: "VARIAZIONI DI LUCE" - Grisey, Lupone}
\cvline{2021}{Partecipazione a Seminari presso Conservatorio Santa Cecilia (RM) tenuti da: 
Giorgio Netti, Giorgio Nottoli, Simone Pappalardo. }

\section{Languages}
\cvline{Italian}{madre lingua}
\cvline{English}{fluent}

\section{Links}
\cvline{1}{Pubblicazioni su Academia: https://conservatoriosantacecilia.academia.edu/LucaSpanedda}
\cvline{2}{Codici/Ricerca: https://github.com/LucaSpanedda/}

\begin{scriptsize}
\vspace{\fill} 
Rome, \today
\end{scriptsize}

\begin{footnotesize}
\begin{Verbatim}
Autorizzo il trattamento dei miei dati personali presenti nel curriculum vitae ai sensi del Decreto Legislativo 30 giugno 2003, n. 196 e del GDPR (Regolamento UE 2016/679).
\end{Verbatim}
\end{footnotesize}


\end{document}