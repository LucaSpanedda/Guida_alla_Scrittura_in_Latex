\documentclass[12pt]{article}
\usepackage{tikz}
\usepackage{pdfpages}
\usepackage{blindtext}
\usepackage{multicol}
\usepackage{lipsum}% http://ctan.org/pkg/lipsum
\usepackage[margin=2cm,landscape,a3paper]{geometry}% http://ctan.org/pkg/geometry
\pagestyle{empty}% Set page style to empty
\makeatletter
\let\ps@plain\ps@empty% Make plain page style equivalent to empty page style
\makeatletter


\begin{document}

% COMANDO PER INCLUDERE UN PDF (RUOTATO)
\includepdf[pages=-,angle=270]{coverdem.pdf}

% SCRITTURA DEL FILE IN A3
\section*{Note}
Deus ex machina è un brano per performer e live electronics che ha come obiettivo quello di rappresentare il 
\emph{logos} da un punto di vista postmoderno.
La ragione di questo brano ha fondo in una volontà critica verso i valori culturali 
e gli assunti fondamentali condivisi dalla società occidentale; 
che è continuamente soggetta ad evidenti cortocircuiti e malfunzionamenti a danno degli individui che la compongono.
\newline
Il cortocircuito, e dunque la retroazione a guadagno infinito che porta all'autodistruzione 
del sistema stesso che la genera,
è il processo centrale su cui si basa il dialogo fra la voce e l'elettronica.
Nel brano, la voce viene processata elettronicamente in modo da collassare su sé stessa, implodere.
\newline
La nascita del verbo, la libertà espressiva che corrisponde ad un'origine della vocalità ricercata dal performer, 
viene dunque continuamente ostracizzata ed ostacolata da questa macchina artificiale che è in realtà il prodotto della memoria della voce stessa.
L'elettronica e la voce sono in un rapporto intrecciato di dualismo dove l'una influenza il comportamento dell'altra.
\newline \newline
Deus Ex Machina è un rituale.

\end{document}
