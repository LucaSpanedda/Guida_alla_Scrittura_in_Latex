\documentclass{article}
\usepackage[utf8]{inputenc}
\usepackage[english]{babel}

% pacchetto Multicolonna
\usepackage{multicol}
% pacchetto Immagine
\usepackage{graphicx} 

% inizio del documento
\begin{document}
% Titolo, Autore, Data.
\title{Prova Disegno}
\author{Luca Spanedda}
\date{\today}
%...
\maketitle
% --------------------------------------------------------------------------------
% --------------------------------------------------------------------------------
% --------------------------------------------------------------------------------
% --------------------------------------------------------------------------------

% all'interno di un codice Latex può esser necessario di scrivere un codice.
% il problema è che il codice può andare in conflitto ed esser interpretato dal linguaggio Latex.
% La modalità per superare tale difficoltà è usare l'ambiente verbatim.
% Tutto quello che viene scritto dentro verbatim viene stampato dal codice così come è
% e non viene interpretato invece dal compilatore di Latex:

\begin{verbatim}

/* -------------------------------------
   SEZIONE OSCILLATORE 1
   ------------------------------------- */

// Codice della Funzione dell'Oscillatore 1 
// richiamato sul process.

// CONTROLLI che vengono richiamati come argomento della funzione: 
// frequency1 = frequenza da 1. a 10000.
// shape1 = forma d'onda da 0. a 1. per la sine 
// (nel VCS3 da 0. a 10.)
// amposcillatore1 = ampiezza della sine da 0. a 1. 
// (nel VCS3 da 0. a 10.)
// ampsaw1 = ampiezza della saw da 0. a 1. (nel VCS3 da 0. a 10.)

// FINE DEL CODICE

\end{verbatim}