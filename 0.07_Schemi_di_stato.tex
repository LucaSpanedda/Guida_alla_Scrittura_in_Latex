\documentclass[12pt]{article}
\usepackage{tikz}
\usetikzlibrary{arrows}
\begin{document}
\begin{figure}
\begin{center}
% con l'ambiente tikz in latek si possono rappresentare disegni e schemi
% di vario tipo.
\begin{tikzpicture}

\section{Schema di stato}
% con node
% qui vado a definire i disegni in questione. In ordine di apparizione:
% [draw,circle] --> disegna un cerchio
% (s1) at (0,0) --> nome dell'oggetto e dove viene posizionato
% {A} --> contenuto del'oggetto in questione.
\begin{figure}[h!]
\begin{center}
\begin{tikzpicture}
\node[draw,circle] (s1) at (-20,-3.5) {in};
\node[draw,circle] (s2) at (-18,-1) {A1};
\node[draw,circle] (s3) at (-14,-1) {B1};
\node[draw,circle] (s4) at (-10,0) {C1};
\node[draw,circle] (s5) at (-20,2) {+};
\node[draw,circle] (s6) at (-14,2) {out 1};
\node[draw,circle] (s7) at (-18,-6) {A2};
\node[draw,circle] (s8) at (-14,-6) {B2};
\node[draw,circle] (s9) at (-10,-7) {C2};
\node[draw,circle] (s10) at (-20,-9) {+};
\node[draw,circle] (s11) at (-14,-9) {out 2};

% ARROWS
% in path vado a definire i collegamenti, ad esempio:
% (s2) edge[->] (s1)
% definisce che l'oggetto (s2) punterà con edge all'oggetto (s1)
\path
    (s1) edge[->] (s2)
    (s2) edge[->] (s3)
    (s3) edge[->] (s4)
    (s4) edge[->] (s5)
    (s2) edge[->] (s5)
    (s5) edge[->] (s6)
    (s1) edge[->] (s7)
    (s7) edge[->] (s8)
    (s8) edge[->] (s9)
    (s9) edge[->] (s10)
    (s7) edge[->] (s10)
    (s10) edge[->] (s11)
    ;
\end{tikzpicture}
% commento del tutto:
\caption{Algoritmo}
\end{center}
\end{figure}

\end{document}
