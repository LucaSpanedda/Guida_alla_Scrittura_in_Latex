\documentclass{article}
\usepackage[utf8]{inputenc}
\usepackage[english]{babel}

% pacchetto Multicolonna
\usepackage{multicol}
% pacchetto Immagine
\usepackage{graphicx} 

% inizio del documento
\begin{document}
% Titolo, Autore, Data.
\title{Analisi della partitura di Sequenza I di Luciano Berio (1958)}
\author{Luca Spanedda}
\date{\today}
%...
\maketitle
% --------------------------------------------------------------------------------
% --------------------------------------------------------------------------------
% --------------------------------------------------------------------------------
% --------------------------------------------------------------------------------

% questo è un ABSTRACT inserito all'inizio di un articolo

\begin{Introduzione}
Sequenza I è il primo brano della serie delle Sequenze di Luciano Berio.
Le sequenze sono dei brani composti per strumento solo. 
Sono essenzialmente brani dal carattere virtuosistico che nascono con il fine di esplorare tecnicamente le possibilità
espressive di un determinato strumento.
\end{abstract}
% ----------------------------------------

% per poi proseguire col resto del testo
\newpage
% sezione del documento multicolonna 
\begin{multicols}{2} % uso {2} colonne
% titolo posto su una singola colonna
[\section{La Notazione Proporzionale}]
Sequenza I (1958) è scritta in notazione proporzionale.
La notazione proporzionale è un tipo di notazione musicale che richiede all’esecutore
di eseguire quello che è rappresentato in partitura tenendo contro del valore
temporale di un determinato spazio definito.
\end{multicols}
% ----------------------------------------

\end{document}