\documentclass{article}
\usepackage[utf8]{inputenc}
\usepackage[english]{babel}

% pacchetto Multicolonna
\usepackage{multicol}
% pacchetto Immagine
\usepackage{graphicx} 

% inizio del documento
\begin{document}
% Titolo, Autore, Data.
\title{Titolo del testo}
\author{Luca Spanedda}
\date{\today}
%...
\maketitle

% questo è un ABSTRACT inserito all'inizio di un articolo

\begin{abstract}
Questo è il testo che comparirà all'interno dell'abstract
\end{abstract}
% ----------------------------------------

% per poi proseguire col resto del testo su una nuova pagina
\cleardoublepage
% sezione del documento multicolonna 
\begin{multicols}{2} 
% uso {2} colonne
% titolo posto su una singola colonna
[\section{La Notazione Proporzionale}]
Sequenza I (1958) è scritta in notazione proporzionale.
La notazione proporzionale è un tipo di notazione musicale che richiede all’esecutore
di eseguire quello che è rappresentato in partitura tenendo contro del valore
temporale di un determinato spazio definito.
\end{multicols}
% ----------------------------------------

\end{document}
